\begin{appendices}
\chapter{Webcam Detection and Classification Source Code}

\begin{lstlisting}[language=Python, caption=Webcam Detection and Classification]

# Import packages
import os
import cv2
import numpy as np
import tensorflow as tf
import sys

# This is needed since the notebook is stored in the object_detection folder.
sys.path.append("..")

# Import utilites
from utils import label_map_util
from utils import visualization_utils as vis_util


# Name of the directory containing the object detection module we're using
MODEL_NAME = 'inference_graph'

# Grab path to current working directory
CWD_PATH = os.getcwd()

# Path to frozen detection graph .pb file, which contains the model that is used
# for object detection.
PATH_TO_CKPT = os.path.join(CWD_PATH,MODEL_NAME,'frozen_inference_graph.pb')

# Path to label map file
PATH_TO_LABELS = os.path.join(CWD_PATH,'training','labelmap.pbtxt')

# Number of classes the object detector can identify
NUM_CLASSES = 16



## Load the label map.
# Label maps map indices to category names, so that when our convolution
# network predicts `5`, we know that this corresponds to `king`.
# Here we use internal utility functions, but anything that returns a
# dictionary mapping integers to appropriate string labels would be fine
label_map = label_map_util.load_labelmap(PATH_TO_LABELS)
categories = label_map_util.convert_label_map_to_categories(label_map, max_num_classes=NUM_CLASSES, use_display_name=True)
category_index = label_map_util.create_category_index(categories)


# Load the Tensorflow model into memory.
detection_graph = tf.Graph()
with detection_graph.as_default():
	od_graph_def = tf.GraphDef()
	with tf.gfile.GFile(PATH_TO_CKPT, 'rb') as fid:
		serialized_graph = fid.read()
		od_graph_def.ParseFromString(serialized_graph)
		tf.import_graph_def(od_graph_def, name='')

	sess = tf.Session(graph=detection_graph)
	
	
# Define input and output tensors (i.e. data) for the object detection classifier

# Input tensor is the image
image_tensor = detection_graph.get_tensor_by_name('image_tensor:0')

# Output tensors are the detection boxes, scores, and classes
# Each box represents a part of the image where a particular object was detected
detection_boxes = detection_graph.get_tensor_by_name('detection_boxes:0')

# Each score represents level of confidence for each of the objects.
# The score is shown on the result image, together with the class label.
detection_scores = detection_graph.get_tensor_by_name('detection_scores:0')
detection_classes = detection_graph.get_tensor_by_name('detection_classes:0')

# Number of objects detected
num_detections = detection_graph.get_tensor_by_name('num_detections:0')


# Initialize webcam feed
video = cv2.VideoCapture(0)
ret = video.set(3,1280)
ret = video.set(4,720)


while(True):

	# Acquire frame and expand frame dimensions to have shape: [1, None, None, 3]
	# i.e. a single-column array, where each item in the column has the pixel RGB value
	ret, frame = video.read()
	frame_expanded = np.expand_dims(frame, axis=0)

	# Perform the actual detection by running the model with the image as input
	(boxes, scores, classes, num) = sess.run(
		[detection_boxes, detection_scores, detection_classes, num_detections],
			feed_dict={image_tensor: frame_expanded})

	# Draw the results of the detection (aka 'visulaize the results')
	vis_util.visualize_boxes_and_labels_on_image_array(
		frame,
		np.squeeze(boxes),
		np.squeeze(classes).astype(np.int32),
		np.squeeze(scores),
		category_index,
		use_normalized_coordinates=True,
		line_thickness=8,
		min_score_thresh=0.60)

	# All the results have been drawn on the frame, so it's time to display it.
	cv2.imshow('Object detector', frame)

	# Press 'q' to quit
	if cv2.waitKey(1) == ord('q'):
		break

# Clean up
video.release()
cv2.destroyAllWindows()


\end{lstlisting}





\end{appendices}





