\chapter{Existing System Study}
\label{chap2}
%##########################################


\vspace*{45 ex}
%##########################################



\paragraph*{Outline:} This chapter presents the following:
\begin{enumerate}
\setlength{\itemsep}{-0.3em}
\item A brief Introduction
\item Study methods 
\item What is Android Operating System
\item Software Requirements
\end{enumerate}

%+++++++++++++++++++++++++++++++++++++++++++++++++++++++++++++++++
\newpage
\section{A brief Introduction}
continued to get smaller in	size, using	less power and performing more advanced calculations.	In 2007	Apple released their iPhone	to achieve the	next goal in computing. This new type of communication tool, called Smartphone, is	generally referred to as a phone, which is a poor labelling. A Smartphone is a handheld computer, which	can place phone	calls. Although	the	term Smartphone	was	first used in 1992, Apple was	the	first company to release a Smartphone to a wider audience. This evolution is led by computer manufacturers and software companies and not handset manufacturers, which have controlled the market thus far.\\

\noindent
One competitor to Apple iPhone OS is the Android OS. Android originates from a small software company, acquired by Google and is now owned by Open Handset Alliance(OHA), where Google is a member. OHA has over a hundred member companies such as mobile operators, semiconductor companies, handset manufacturers, software companies and commercialisation companies. Driven through the Apache License, anyone can use the Android Software Development Kit(SDK) to develop applications to run on the Android OS. Especially interesting for Android is it's used of common non-proprietary techniques such as the Java programming language in combination with Extensible Markup Language(XML). This makes it open, simple and easy to use for a substantial part of the developer community.

\noindent
The thesis reflects experiences of using XML, the Java Programming Language and Android SDK. Questions answered are the following:
\begin{itemize}
\item {What is Android and how does it work using Linux Kernel, XML and Java Programming Language? }
\item {How does a developer install an Android application on an Android device, without using the Android market? }
\item {What does a developer do to connect the Android application to Facebook Connect? }
\item {How does a developer connect, and synchronize data, with an internet-connected server? }
\end{itemize}

\noindent
This thesis focuses on development of Android applications to be used on a handheld device running on the Android OS. The thesis covers the basics such as setting up a development environment, downloading appropriate tools and add-ons and shows how to get developers working with Android application development. In essence this thesis serves as a guideline to intermediate developers, seeking solutions to practical problems, who might find current literature a bit too shallow on giving answers to these questions, such as “How about the warning messages I get when compiling Android applications” and “The AVD does not start when I use API level less than 16”.

\section{Study methods}
This is an empirical qualitative study, based on reading above mentioned literature and testing their examples. Tests are made by programming according to books and online resources, with the explicit goal to find best practices and a more advanced understanding of Android. One use case is presented in this thesis as a “Hello World!” application, explaining what happens behind the scenes. The other use case is a developed application which is presented at a(conceptual level.
\section{What is Android Operating System}
“Android was built from the ground up with the explicit goal to be the first open, complete, and free platform created specifically for mobile devices.”\\

\bigskip
- Ableson F. et al, Unlocking Android, page 4.\\

\noindent
Android is an open system, and is free to use by anyone. A handset manufacturer can use Android if they follow the agreement stated in the Software Development Kit. There are no restrictions or( requirement for the handset manufacturer to share their extensions with anyone else, as there are in other open source software, if they leave the Linux kernel as is. The Linux kernel is under a different and more restricted license than Android.\\

\noindent
Android is a software environment and not a hardware platform, which includes an OS, built on Linux kernel-based OS hosting the Dalvik virtual machine. The Dalvik virtual machine runs Android applications as instances of the virtual machine. Android contains a rich user interface, application framework, Java Class libraries and multimedia support. Android also comes with built-in applications containing features such as short message service functionality messaging, phone capabilities and an address book(contacts).

\section{Software Requirements}
This Project development requires the knowledge of following Tools (Open-Source) and Artifical Programming \& Scripting Languages :—
\bigskip
\begin{enumerate}
	\setlength{\itemsep}{-0.3em}
	\item Android Studio IDE
	\item Text-Editor
	\item Virtual Device (i.e emulater) 
	\item Firebase Server Side Database
		\begin{itemize}
			\item Authentication
			\item Real Time Database
			\item Cloud Storage
			\item Storage
			\item Cloud Functions
		\end{itemize}
	\item SqlLite Client Side Database
	\item Node JS Environment
	\item JAVA ME, XML,JavaScript, Node JS for Firebase Functions \& Others languages.
	\item Other basic requirements. 
\end{enumerate}


%\section{Summary}
%In this chapter, we describe .......    
