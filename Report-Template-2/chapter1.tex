\chapter*{Chapter 1}
\addcontentsline{toc}{chapter}{\numberline{}Chapter 1}
\section{Introduction}
continued to get smaller in	size, using	less power and performing more advanced calculations.	In 2007	Apple released their iPhone	to achieve the	next goal in computing. This new type of communication tool, called Smartphone, is	generally referred to as a phone, which is a poor labelling. A Smartphone is a handheld computer, which	can place phone	calls. Although	the	term Smartphone	was	first used in 1992, Apple was	the	first company to release a Smartphone to a wider audience. This evolution is led by computer manufacturers and software companies and not handset manufacturers, which have controlled the market thus far.\\

\noindent
One competitor to Apple iPhone OS is the Android OS. Android originates from a small software company, acquired by Google and is now owned by Open Handset Alliance(OHA), where Google is a member. OHA has over a hundred member companies such as mobile operators, semiconductor companies, handset manufacturers, software companies and commercialisation companies. Driven through the Apache License, anyone can use the Android Software Development Kit(SDK) to develop applications to run on the Android OS. Especially interesting for Android is it's used of common non-proprietary techniques such as the Java programming language in combination with Extensible(Markup(Language((XML). This makes it open, simple and easy to use for a substantial part of the developer community.

\noindent
The thesis reflects experiences of using XML, the Java Programming Language and Android SDK. Questions answered are the following:
\begin{itemize}
\item {What is Android and how does it work using Linux Kernel, XML and Java Programming Language? }
\item {How does a developer install an Android application on an Android device, without using the Android market? }
\item {What does a developer do to connect the Android application to Facebook Connect? }
\item {How does a developer connect, and synchronize data, with an internet-connected server? }
\end{itemize}

\noindent
This thesis focuses on development of Android applications to be used on a handheld device running on the Android OS. The thesis covers the basics such as setting up a development environment, downloading appropriate tools and add-ons and shows how to get developers working with Android application development. In essence this thesis serves as a guideline to intermediate developers, seeking solutions to practical problems, who might find current literature a bit too shallow on giving answers to these questions, such as “How about the warning messages I get when compiling Android applications” and “The AVD does not start when I use a SD-card size of 2MiB”.